\documentclass[12pt]{article}
\usepackage[utf8]{inputenc}
\usepackage{float}
\usepackage{amsmath}

\usepackage{tikz}
\usepackage[hmargin=3cm,vmargin=6.0cm]{geometry}
\topmargin=-2cm
\addtolength{\textheight}{6.5cm}
\addtolength{\textwidth}{2.0cm}
\setlength{\oddsidemargin}{0.0cm}
\setlength{\evensidemargin}{0.0cm}
\usepackage{indentfirst}
\usepackage{amsfonts}

\begin{document}

\section*{Student Information}

Name : Deniz Karakoyun\\

ID : 2580678\\


\section*{Answer 1}
\subsection*{a)} 
To find the value of \( k \), we first need to ensure that the joint probability density function \( f_{X,Y}(x, y) \) satisfies the properties of a probability density function, which means it must integrate to 1 over the entire range of possible values for \( x \) and \( y \).

Given:
\[ f_{X,Y}(x, y) =
\begin{cases}
x + ky^3 & \text{if } 0 \leq x \leq 1, \ 0 \leq y \leq 1 \\
0 & \text{otherwise}
\end{cases}
\]

We have to integrate \( f_{X,Y}(x, y) \) over the range \( 0 \leq x \leq 1 \) and \( 0 \leq y \leq 1 \) and equate it to 1 to find the value of \( k \).

\[ \int_{0}^{1} \int_{0}^{1} (x + ky^3) \, dy \, dx = 1 \]

\[ \int_{0}^{1} \left[ xy + \frac{k}{4}y^4 \right]_{0}^{1} \, dx = 1 \]

\[ \int_{0}^{1} \left( x + \frac{k}{4} \right) \, dx = 1 \]

\[ \left[ \frac{x^2}{2} + \frac{k}{4}x \right]_{0}^{1} = 1 \]

\[ \frac{1}{2} + \frac{k}{4} = 1 \]

\[ \frac{1}{2} + \frac{k}{4} = 1 \]

\[ \frac{k}{4} = \frac{1}{2} \]

\[ k = 2 \]

\subsection*{b)} 
To find \( P(X = \frac{1}{2}) \), we integrate \( f_{X,Y}(x, y) \) with respect to \( y \) from \( 0 \) to \( 1 \), while setting the borders of x  from \( \frac{1}{2} \) to \( \frac{1}{2} \).

\[ \int_{0}^{1} \int_{\frac{1}{2}}^{\frac{1}{2}} (x + 2y^3) \, dx \, dy = 0 \] since the integral with the same bound equals to zero 

\[ P(X = \frac{1}{2}) = \int_{0}^{1} 0 \, dy \]

\[ = 0 \]

For all continuous variables, the probability mass function (pmf) is always equal to zero and X is a continuous random variable so its probability at one point is always 0. From here we can conclude that \( P(X = \frac{1}{2}) \) = 0.

\subsection*{c)} 
To find \( P(0 \leq X \leq \frac{1}{2}, 0 \leq Y \leq \frac{1}{2}) \), we integrate \( f_{X,Y}(x, y) \) over the region defined by \( 0 \leq x \leq \frac{1}{2} \) and \( 0 \leq y \leq \frac{1}{2} \).

\[ P(0 \leq X \leq \frac{1}{2}, 0 \leq Y \leq \frac{1}{2}) = \int_{0}^{\frac{1}{2}} \int_{0}^{\frac{1}{2}} (x + 2y^3) \, dy \, dx \]

\[ = \int_{0}^{\frac{1}{2}} ( xy + \frac{2}{4}y^4 )  \Biggr|_{0}^{\frac{1}{2}} \, dx \] 

\[ = \int_{0}^{\frac{1}{2}} \left( \frac{x}{2} + \frac{1}{32} \right) \, dx \]

\[ =( \frac{x^2}{4} + \frac{x}{32} ) \Biggr|_{0}^{\frac{1}{2}} \]

\[ = \left( \frac{1}{16} + \frac{1}{64} \right) - \left( \frac{0^2}{4} + \frac{0}{32} \right) \]



\[ P(0 \leq X \leq \frac{1}{2}, 0 \leq Y \leq \frac{1}{2}) = \frac{5}{64} \]




\section*{Answer 2}
\subsection*{a)} 
To find the marginal PDF of \( Y \), we need to integrate the joint PDF \( f_{X,Y}(x, y) \) with respect to \( x \) over the range \( 0 < x < \infty \).

Given:
\[ f_{X,Y}(x, y) = \frac{e^{-y - \frac{x}{y}}}{y} \]

To find the marginal PDF of \( Y \), we integrate \( f_{X,Y}(x, y) \) with respect to \( x \) from \( 0 \) to \( \infty \), and simplify:

\[ f_{Y}(y) = \int_{0}^{\infty} \frac{e^{-y - \frac{x}{y}}}{y} \, dx \]

\[ f_{Y}(y) = \frac{1}{y} \int_{0}^{\infty} e^{-y - \frac{x}{y}} \, dx \]

\[ f_{Y}(y) = \frac{1}{y} \int_{0}^{\infty} e^{-y} e^{-\frac{x}{y}} \, dx \]

\[ f_{Y}(y) = \frac{1}{y} e^{-y} \int_{0}^{\infty} e^{-\frac{x}{y}} \, dx \]

\[ f_{Y}(y) = \frac{1}{y} e^{-y} \left[ -y e^{-\frac{x}{y}} \right]_{0}^{\infty} \]

\[ f_{Y}(y) = \frac{1}{y} e^{-y} \left( 0 - (-y) \right) \]

\[ f_{Y}(y) = e^{-y} \]

Therefore, the marginal PDF of \( Y \) is \( f_{Y}(y) = e^{-y} \) for \( 0 < y < \infty \). \( Y \) follows an exponential distribution with \( \lambda = 1 \)  where  the pdf of exponential distribution is \( \lambda \cdot e^{-\lambda x} \).

\subsection*{b)} 

We know the distribution of Y so its expected value(the mean of the exponential distribution) can be easily found. Expected value formula for exponential distribution is \(\frac{1}{\lambda} \) so if we substitute \( \lambda = 1 \) we can find \[ E(Y) = \mu_y = \frac{1}{\lambda} = 1 \]  


So, the expected value of \( Y \) is 1.


\section*{Answer 3}
\subsection*{a)} 
 To find the probability that at least 9\% of the soldiers belong to the naval forces among 1000 soldiers, we approximate the binomial distribution to a normal distribution with parameters:
\[
\mu = n\cdot p = 1000 \times 0.1 = 100
\]
\[
\sigma =\sqrt{ n\cdot p\cdot (1 - p) } =  \sqrt{1000 \times 0.1 \times 0.9} = \sqrt{90} = 9.49
\]

Applying the Central Limit Theorem with continuity correction, we want to find \(P(x \geq 90)\), which is equivalent to \(1 - P(x < 89.5)\). This can be expressed as \(1 - P\left(\frac{x - 100}{9.49} < \frac{89.5 - 100}{9.49}\right) = 1 - P(z < -1.106)\), where \(z\) is a standard normal random variable. Using the standard normal distribution table, \(\Phi(-1.106) = 0.134\). Therefore, \(1 - 0.134 = 0.866\).




\subsection*{b)} 
 For 2000 soldiers, we repeat the same steps with \(n = 2000\):
\[
\mu = n\cdot p  =2000 \times 0.1 = 200
\]
\[
\sigma = \sqrt{ n\cdot p\cdot (1 - p) } = \sqrt{2000 \times 0.1 \times 0.9} = \sqrt{180} = 13.416
\]

The minimum number of soldiers belonging to naval forces is \(2000 \times 0.09 = 180\). Thus, we want to find \(P(x \geq 180) = 1 - P(x < 179.5) = 1 - P\left(\frac{x - 200}{13.416} < \frac{179.5 - 200}{13.416}\right) = 1 - P(z < -1.528)\). Using the standard normal distribution table, \(\Phi(-1.528) = 0.063\). Therefore, \(1 - 0.063 = 0.937\).

The probability increased when the sample size increased to 2000 soldiers because as the sample size increases, the sample mean approaches the population mean, resulting in a more accurate approximation. Thus, the probability of having at least 9\% naval forces increased.
\section*{Answer 4}
\subsection*{a)} 
To calculate the probability that a randomly selected elephant will live more than 60 years but less than 75 years, we need to find the area under the normal distribution curve between these two ages.

Given:
Mean (\( \mu \)) = 65 years
Standard deviation (\( \sigma \)) = 6 years

We want to find:
\[ P(60 < X < 75) \]

To do this, we first standardize the values using the z-score formula:
\[ z = \frac{x - \mu}{\sigma} \]

For \( x = 60 \):
\[ z_1 = \frac{60 - 65}{6} = -\frac{5}{6} \]

For \( x = 75 \):
\[ z_2 = \frac{75 - 65}{6} = \frac{10}{6} = \frac{5}{3} \]

Now, we look up the corresponding probabilities in the standard normal distribution table:
\[ P(-\frac{5}{6} < Z < \frac{5}{3}) \]

Subtracting the cumulative probability corresponding to \( z = -\frac{5}{6} \) from the cumulative probability corresponding to \( z = \frac{5}{3} \) gives us the desired probability.

\(\Phi(1.67) - \Phi(-0.83)  = 0.953 - 0.203 = 0.750\)

\subsection*{b)} 
Here is my MATLAB code for part b:\\
\begin{verbatim}
mu = 65;
sigma = 6;
sample_size = [20, 100, 1000];

figure;
for i = 1:length(sample_size)
    sample = normrnd(mu, sigma, sample_size(i), 1);
    
    subplot(1, 3, i);
    histogram(sample, 20);
    title(['Sample Size: ', num2str(sample_size(i))]);
    xlabel('Lifespan (years)');
    ylabel('Frequency');
end
\end{verbatim}

You can see the figure in the next page Figure 1
\begin{figure}[htbp]
    \centering
    \hspace{-2.5cm}\includegraphics[scale=0.5]{../../../../../Downloads/denmat.jpg}
    \caption{For part b) Frequency / Lifespan Histogram}
    \label{fig:your_image}
\end{figure}

\subsection*{c)} 

 In the 891 iterations of the code, it was observed that in at least 70\% of cases, elephants had a lifespan between 60 and 75 years in the last two lines of the code. However, in only 10 iterations, this percentage was at least 85\%. This discrepancy can be explained by the findings of part a. In part a, it was determined that the probability for a single elephant to have a lifespan between 60 and 75 years is 75\%. Therefore, when the observed percentage is at least 70\%, it aligns closely with the probability for a randomly chosen single elephant. However, when the observed percentage exceeds 85\%, it deviates significantly from this probability, indicating that such occurrences are rare in the simulation.\\

Here is my MATLAB code for part c:\\

\begin{verbatim}
mu = 65;
sigma = 6;
sample_size = 100;
iteration = 1000;

count_60_75 = zeros(iteration, 1);
count_70percent = 0;
count_85percent = 0;

for i = 1:iteration
    lifespan = normrnd(mu, sigma, sample_size, 1);
    
    count_60_75(i) = sum(lifespan >= 60 & lifespan <= 75);
    if count_60_75(i) >= 0.7 * sample_size
        count_70percent = count_70percent + 1;
    end
    
    if count_60_75(i) >= 0.85 * sample_size
        count_85percent = count_85percent + 1;
    end
end

fprintf('Number of simulations where at least 70\%\% of elephants had a lifespan 
between 60 and 75 years: \%d\n', count_70percent);
fprintf('Number of simulations where at least 85\%\% of elephants had a lifespan 
between 60 and 75 years: \%d\n', count_85percent);

>>Number of simulations where at least 70\% of elephants had a lifespan between 
  60 and 75 years: 891
>>Number of simulations where at least 85\% of elephants had a lifespan between 
  60 and 75 years: 10
\end{verbatim}

\end{document}
